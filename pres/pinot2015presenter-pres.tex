\documentclass[table]{beamer}
%%\documentclass[handout]{beamer}

\mode<presentation>
{
  \useoutertheme[subsection=false]{miniframes}
  %%\useoutertheme[footline=authortitle]{miniframes}
  \usecolortheme[RGB={233,37,73}]{structure}
  %%\usecolortheme[RGB={255,121,78}]{structure}
  \usecolortheme{dolphin}
  \usecolortheme{orchid}
  \useinnertheme{circles}
  \setbeamerfont{block title}{size=\normalsize}
  \setbeamercovered{transparent}

  %%% le foot pour avoir la numérotation des slides %%%
  \setbeamertemplate{footline}{%
    \leavevmode%
    \hbox{%
      \begin{beamercolorbox}[wd=.5\paperwidth,ht=2.5ex,dp=1.125ex,
        leftskip=.3cm plus1fill,rightskip=.3cm]{author in head/foot}%
        \usebeamerfont{title in head/foot}\insertshorttitle
      \end{beamercolorbox}%
      \begin{beamercolorbox}[wd=.5\paperwidth,ht=2.5ex,dp=1.125ex,
        leftskip=.3cm,rightskip=.3cm plus1fil]{title in head/foot}%
        \usebeamerfont{author in head/foot}\insertshortauthor\hfill
        \insertframenumber/\inserttotalframenumber
      \end{beamercolorbox}%
    }%
    \vskip0pt%
  }
}

\mode<handout>
{
  \usepackage{pgfpages}
  \pgfpagesuselayout{4 on 1}[a4paper,border shrink=5mm,landscape]
}

\usepackage[utf8]{inputenc}
\usepackage{lmodern}
\usepackage[T1]{fontenc}
\usepackage[english,francais]{babel}

%% jour ouvert (ou open) pour les calendriers
\newcommand{\+}{\cellcolor[gray]{1}\bfseries}
%% exception
\newcommand{\x}{\cellcolor[gray]{0.8}\rmfamily\itshape}

\def\S{\mathcal{S}}

\newcommand{\nologo}{\setbeamertemplate{logo}{}}

\newenvironment{foreignpar}[1][english]{%
    \em\selectlanguage{#1}%
}{}
\newcommand*{\foreign}[2][english]{%
    \emph{\foreignlanguage{#1}{#2}}%
}

\title[Présenter une~liste~de~dates de~manière~lisible]
{Présenter une~liste~de~dates de~manière~lisible~:
  complexité~et~algorithme}

\author{Guillaume Pinot}

\institute[Canal TP] % (optional, but mostly needed)
{
  Canal TP\\
  20 rue Hector Malot\\
  75012 Paris, France\\
  \texttt{guillaume.pinot@canaltp.fr}}
%% - Use the \inst command only if there are several affiliations.
%% - Keep it simple, no one is interested in your street address.

\date{ROADEF 2015}
%% - Either use conference name or its abbreviation.
%% - Not really informative to the audience, more for people (including
%%   yourself) who are reading the slides online

%% If you have a file called "university-logo-filename.xxx", where xxx
%% is a graphic format that can be processed by latex or pdflatex,
%% resp., then you can add a logo as follows:
\pgfdeclareimage[width=.2\linewidth]{logo}{images/canaltp}
\logo{\pgfuseimage{logo}}


%% Delete this, if you do not want the table of contents to pop up at
%% the beginning of each subsection:
\AtBeginSection[]
{
  \begin{frame}<beamer>
    \frametitle{Table des matières}
    \tableofcontents[currentsection,hideothersubsections]
  \end{frame}
}
\AtBeginSubsection[]
{
  \begin{frame}<beamer>
    \frametitle{Table des matières}
    \tableofcontents[currentsection,subsectionstyle=show/shaded/hide]
  \end{frame}
}


\begin{document}

\begin{frame}
  \titlepage    
\end{frame}

\section*{Introduction}

\begin{frame}
  \frametitle{Présentation de Canal TP}

  Canal TP:
  \begin{itemize}
  \item éditeur logiciel fournissant expertise et solutions;
  \item centré sur l'information voyageurs.
  \end{itemize}

  \emph{navitia}:
  \begin{itemize}
  \item un moteur disponible en logiciel lible
    \url{https://github.com/CanalTP/navitia};
  \item un \foreign{web service} \url{http://navitia.io};
  \item présentation de l'offre de transport: lignes, arrêts, horaires;
  \item autocompletion et géolocalisation;
  \item calcul d'itinéraires multimodaux porte-à-porte.
  \end{itemize}
\end{frame}

\begin{frame}
  \frametitle{Problématique}

  \begin{block}{Calendrier}
    \centering
    \begin{tabular}{|ccccccc|}
      \hline
      Lu  & Ma & Me & Je & Ve & Sa & Di\\
      \hline
      \+27&\+28&\+29&\+30&  1 &  2 &  3\\
      \+4 & \+5& \+6& \+7&  8 &  9 & 10\\
      \+11&\+12&\+13& 14 &\+15& 16 & 17\\
      \+18&\+19&\+20&\+21&\+22& 23 & 24\\
      25  &\+26&\+27&\+28&\+29&\+30& 31\\
      \hline
    \end{tabular}
  \end{block}

  \begin{block}{Sortie}
    \centering 11110001111000111010011111000111110
  \end{block}

  Et pour un humain?
\end{frame}

\begin{frame}%%[allowframebreaks]
  \frametitle{Table des matières}
  \tableofcontents[hideallsubsections]
  %% You might wish to add the option [pausesections]
\end{frame}

\section{Le problème}

{\nologo
\begin{frame}
  \frametitle{Description du problème}

  \begin{block}{Calendrier}
    \centering
    \begin{tabular}{|ccccccc|}
      \hline
      Lu  & Ma & Me & Je & Ve & Sa & Di\\
      \hline
      \+27&\+28&\+29&\+30&\x 1&  2 &  3\\
      \+4 & \+5& \+6& \+7&\x 8&  9 & 10\\
      \+11&\+12&\+13&\x 14&\+15& 16 & 17\\
      \+18&\+19&\+20&\+21&\+22& 23 & 24\\
      \x 25&\+26&\+27&\+28&\+29&\+\x 30& 31\\
      \hline
    \end{tabular}
  \end{block}

  \begin{block}{Sortie humaine}
    \centering
    \begin{tabular}{rl}
      Du Lundi au Vendredi         & \visible<2->{un rythme hebdomadaire}\\
      du 27 avril au 31 mai 2015   & \visible<2->{une période de validité}\\
      sauf les 1, 8, 14 et 25 mai, & \visible<2->{une liste de date exclues}\\
      avec en plus le 30 mai.      & \visible<2->{une liste de date incluses}\\
      \visible<3->{Objectif:}      & \visible<3->{minimiser le nombre d'exceptions}\\
    \end{tabular}
  \end{block}
\end{frame}
}

\begin{frame}
  \frametitle{Données du problème}

  \begin{block}{Calendrier}
    \centering
    \begin{tabular}{|ccccccc|}
      \hline
      Lu  & Ma & Me & Je & Ve & Sa & Di\\
      \hline
      \+27&\+28&\+29&\+30&  1 &  2 &  3\\
      \+4 & \+5& \+6& \+7&  8 &  9 & 10\\
      \+11&\+12&\+13& 14 &\+15& 16 & 17\\
      \+18&\+19&\+20&\+21&\+22& 23 & 24\\
      25  &\+26&\+27&\+28&\+29&\+30& 31\\
      \hline
    \end{tabular}
  \end{block}

  \begin{block}{Données du problème}
    \centering\strut
    \begin{math}
      \only<2->{\S = }%
      1111000\only<2->{, }1111000\only<2->{, }%
      1110100\only<2->{, }1111100\only<2->{, }0111110
    \end{math}
  \end{block}

  \pause[3] Les rythmes hebdomadaires de chaque semaine.
\end{frame}

\begin{frame}
  \frametitle{Distance de Hamming}

  \cite{hamming1950error} 
\end{frame}

\begin{frame}
  \frametitle{Modèle mathématique}

  Soit $\S = s_1, s_2, \ldots, s_n$, $n$ chaînes binaires de
  longueur 7.

  Soit
  \begin{math}
    \displaystyle
    d^\Sigma_{s_i} = \sum_{s_j\in\S} d(s_i, s_j)
  \end{math}.

  Le meilleur rythme hebdomadaire est représenté par la chaîne $s$ tel
  que $d^\Sigma_s$ soit minimale.
\end{frame}

\section{Comparaison avec l'existant et complexité}

\begin{frame}
  \frametitle{Le \foreign{closest string problem}}

  \begin{block}{Le \foreign{closest string problem}}
    Soit $\S = s_1, s_2, \ldots, s_n$, $n$ chaînes sur l'alphabet
    $\Sigma$ de longueur $m$.

    Soit $\displaystyle d^{\max}_{s_i} = \max_{s_j\in\S} d(s_i, s_j)$.

    Le \emph{closest string problem} a pour solution la chaîne~$s$ tel
    que $d^{\max}_s$ soit minimale.
  \end{block}

  Complexité:
  \begin{itemize}
  \item NP-difficile \cite{lanctot2003distinguishing}
  \item $O(nm + nd^{\max}_s(16|\Sigma|)^{d^{\max}_s})$
    \cite{ma2008more} (exponentielle sur $n$)
  \end{itemize}
\end{frame}

{\nologo
\begin{frame}
  \frametitle{Comparaison des deux problèmes}

  \begin{block}{Notre problème}
    Soit $\S = s_1, s_2, \ldots, s_n$, $n$ chaînes \alert<2>{binaires}
    de \alert<3>{longueur 7}.

    Soit
    \begin{math}
      \displaystyle
      d^\Sigma_{s_i} = \alert<4>{\sum_{s_j\in\S}} d(s_i, s_j)
    \end{math}.

    Le meilleur rythme hebdomadaire est représenté par la chaîne $s$
    tel que $d^\Sigma_s$ soit minimale.
  \end{block}

  \begin{block}{Le \foreign{closest string problem}}
    Soit $\S = s_1, s_2, \ldots, s_n$, $n$ chaînes \alert<2>{sur
      l'alphabet $\Sigma$} de \alert<3>{longueur $m$}.

    Soit
    \begin{math}
      \displaystyle
      d^{\max}_{s_i} = \alert<4>{\max_{s_j\in\S}} d(s_i, s_j)
    \end{math}.

    Le \emph{closest string problem} a pour solution la chaîne~$s$ tel
    que $d^{\max}_s$ soit minimale.
  \end{block}

  \pause[5] Comme le \foreign{closest string problem} est exponentiel
  sur $n$, nous pouvons espérer trouver un algorithme polynomial à
  notre problème.
\end{frame}
}

\section{Algorithme}

\begin{frame}
  \frametitle{Algorithme}

\end{frame}

\section{Conclusions et perspectives}

\begin{frame}
  \frametitle{Conclusion et perspective}

\end{frame}

\begin{frame}
  \titlepage
\end{frame}

\appendix

\begin{frame}[allowframebreaks]
  \frametitle{Bibliographie}
  \bibliographystyle{apalike}
  \bibliography{pinot2015presenter}
\end{frame}

\end{document}
